\section{Resultados e Análise dos Resultados}
Esta seção apresenta os resultados obtidos experimentalmente no levantamento das curvas características do diodo BY127 (Procedimento 2a) e na análise temporal das cinco montagens retificadoras submetidas a sinal senoidal (Procedimento 2b). São discutidos tanto os aspectos quantitativos quanto qualitativos, relacionando-se os fenômenos observados com os modelos teóricos estudados.

\subsection{Procedimento 2a: Curvas Características $I_D \times V_D$}

A Figura~\ref{fig:arranjo2a} apresenta o arranjo experimental utilizado para o levantamento da curva característica do diodo BY127, incluindo a fonte DC, multímetros e conexões sobre a protoboard. A Figura~\ref{fig:arranjo2a_proto} mostra a protoboard após conclusão da montagem.

\begin{figure}[h]
\centering
\includegraphics[width=0.9\linewidth]{fotos/reduced_arranjo_2a.jpeg}
\caption{Arranjo experimental utilizado no Procedimento 2a.}
\label{fig:arranjo2a}
\end{figure}

\begin{figure}[h]
\centering
\includegraphics[width=0.9\linewidth]{fotos/reduced_arranjo_2a_proto.jpeg}
\caption{Vista da protoboard após montagem do arranjo do Procedimento 2a.}
\label{fig:arranjo2a_proto}
\end{figure}

Os dados experimentais foram obtidos em incrementos de tensão de aproximadamente \(0{,}1~\mathrm{V}\), sendo a corrente medida diretamente no circuito. %Os valores foram fornecidos em arquivo separado (\textit{dados\_exp4.csv}) e utilizados para construção das curvas apresentadas a seguir.

%\noindent\textit{A. Curva característica $I_D\times V_D$.}
\subsubsection{a) Polarização direta (2a1)}

%\hspace{1cm}A execução da Parte A do procedimento permitiu a coleta dos dados necessários para o levantamento da curva característica do diodo. Os valores medidos estão compilados na Tabela I, e a curva $I_D \times V_D$ resultante é apresentada graficamente na Figura 3.
A Tabela~\ref{tab:direta} apresenta os valores experimentais de tensão e corrente obtidos no levantamento da curva característica do diodo BY127.
% \vspace{-.25cm}

\begin{table}[!h]
    \centering
    \caption{Medidas Realizadas para Levantar a Curva}
    \label{tab:medidas}
    \vspace{0.25cm}
    \begin{tabular}{ccc}
        \hline
        \rule{0pt}{3ex} $ V_D \,\,[V]$ & & $ I_D \,\, [mA]$ \\[5pt]
        \hline
        \rule{0pt}{3ex}-0.8 & & 0 \\
        %-0.7 & & 0 \\
        %-0.6 & & 0 \\
        %-0.5 & & 0 \\
        %-0.4 & & 0 \\
        %-0.3 & & 0 \\
        %-0.2 & & 0 \\
        %-0.1 & & 0 \\
        0 & & 0 \\
        0.1 & & 0 \\
        0.2 & & 0 \\
        0.3 & & 0 \\
        0.4 & & 0 \\
        0.5 & & 0.7 \\
        0.6 & & 3.3 \\
        0.7 & & 29.7 \\
        %0.77 & & 30.4 \\[5pt]
        0.71 & & 30.0 \\[5pt]
        \hline
    \end{tabular}
    \label{tab:direta}
\end{table}

A Figura~\ref{fig:curvaIV} mostra a relação exponencial entre a corrente e a tensão direta, evidenciando o início da condução aproximadamente em $0{,}5~\mathrm{V}$.

\begin{figure*}[h]
\centering
\includegraphics[width=0.9\linewidth]{imagens/PNG/reduced_exp4_curvaID.png}
\caption{Curva característica experimental $I_D \times V_D$ do diodo BY127.}
\label{fig:curvaIV}
\end{figure*}

A análise da Tabela~\ref{tab:direta} revela que o diodo inicia sua condução significativa a partir de aproximadamente:
\[
V_{\mathrm{TH}} \approx 0{,}5~\mathrm{V},
\]
valor compatível com diodos de silício de junção retificadora. Acima desse ponto, observa-se crescimento exponencial da corrente, conforme descrito pela equação de Shockley.

A curva resultante apresenta a região não condutora, a região de transição e a região condutora, reproduzindo adequadamente o comportamento teórico esperado.
%\vspace{-.25cm}

%\begin{figure}[h]
%    \centering
%    \includegraphics[width=8cm]{imagens/grafico_ID_x_VD.png}
%    \caption{Curva característica $V_D\times I_D$}
%    \label{fig:VD_ID}
%\end{figure}

\subsubsection{b) Polarização reversa (2a2)}

A Tabela~\ref{tab:reversa} apresenta os dados medidos com o diodo invertido.

\begin{table}[h]
\centering
\caption{Dados experimentais da polarização reversa (Procedimento 2a2).}
\label{tab:reversa}
\begin{tabular}{c c}
\hline
\textbf{$V_D$ (V)} & \textbf{$I_D$ (mA)} \\
\hline
% Insira aqui a tabela completa conforme o arquivo
% Exemplo:
0.1 & 0.0 \\
0.2 & 0.0 \\
0.5 & 0.0 \\
1.0 & 0.0 \\
2.0 & 0.0 \\
3.0 & 0.0 \\
5.0 & 0.0 \\
10.0 & 0.0 \\
20.0 & 0.0 \\
30.0 & 0.0 \\
40.0 & 0.0 \\
50.0 & 0.0 \\
\hline
\end{tabular}
\end{table}

Em todos os pontos medidos, a corrente reversa permaneceu essencialmente nula, indicando que o diodo operou muito abaixo da tensão máxima reversa permitida (\(800~\mathrm{V}\)). %Este resultado confirma o regime de saturação reversa extremamente baixo característico de diodos retificadores de potência.

\subsection{Procedimento 2b: Análise Temporal das Montagens Senoidais}

Nesta etapa, analisou-se a forma de onda sobre o resistor de carga para cada uma das cinco configurações de circuitos. A tensão aplicada foi senoidal, com amplitude de \(8~\mathrm{V_{pp}}\) e frequência de \(100~\mathrm{Hz}\). As figuras abaixo apresentam as formas de onda obtidas no osciloscópio.%espaço reservado para inclusão das fotografias do osciloscópio.

\subsubsection{Circuito 1: Diodo em série (Retificação de meia onda)}

A saída corresponde apenas ao semiciclo positivo da senoide, sendo o semiciclo negativo completamente bloqueado. A forma de onda apresenta valor de pico reduzido devido à queda direta do diodo.

%\textbf{Espaço reservado para foto do osciloscópio:}


\begin{figure}[h]
\centering
\fbox{\begin{minipage}{0.9\linewidth}
\includegraphics[scale=0.133]{fotos/reduced_2.mat4.jpg}
%\vspace{2cm}
%\centering\textit{Inserir aqui a foto correspondente ao Circuito 1.}
%\vspace{2cm}
\end{minipage}}
\caption{Forma de onda do Circuito 1 (meia onda).}
\end{figure}

\textbf{Atraso de condução medido:}
\[
t_{\mathrm{delay,1}} = \underline{\hspace{2cm}}~\mathrm{ms}
\]

\subsubsection{Circuito 2: Dois diodos em série}

Com dois diodos em série, a queda total de condução é aproximadamente:
\[
V_{\gamma} \approx 2 \times 0{,}7~\mathrm{V} \approx 1{,}4~\mathrm{V}.
\]

A forma de onda é similar à do circuito 1, porém a condução inicia apenas quando a tensão da fonte supera o valor acima.

%\textbf{Espaço reservado para foto:}
\begin{figure}[h]
\centering
\fbox{\begin{minipage}{0.9\linewidth}
%\vspace{2cm}
\includegraphics[scale=0.133]{fotos/reduced_8.jpg}
%\centering\textit{Foto do Circuito 2.}
%\vspace{2cm}
\end{minipage}}
\caption{Forma de onda do Circuito 2.}
\end{figure}

\[
t_{\mathrm{delay,2}} = \underline{\hspace{2cm}}~\mathrm{ms}
\]

\subsubsection{Circuito 3: Dois diodos em antissérie}

Nesta configuração, cada semiciclo é bloqueado por um dos diodos, resultando em praticamente nenhuma condução. Apenas pequenos pulsos podem ser observados devido à capacitância de junção.

%\textbf{Espaço reservado para foto:}
\begin{figure}[h]
\centering
\fbox{\begin{minipage}{0.9\linewidth}
%\vspace{2cm}
\includegraphics[scale=0.133]{fotos/reduced_3.mat4.jpg}
%\centering\textit{Foto do Circuito 2.}
%\vspace{2cm}
\end{minipage}}
\caption{Forma de onda do Circuito 3.}
\end{figure}

\[
t_{\mathrm{delay,3}} = \underline{\hspace{2cm}}~\mathrm{ms}
\]

\subsubsection{Circuito 4: Dois diodos em antiparalelo}

Um diodo conduz no semiciclo positivo e o outro conduz no semiciclo negativo. A forma de onda resultante apresenta ambos os semiciclos, porém recortados no início devido à queda \(V_{\gamma}\).

%\textbf{Espaço reservado para foto:}
\begin{figure}[h]
\centering
\fbox{\begin{minipage}{0.9\linewidth}
%\vspace{2cm}
\includegraphics[scale=0.133]{fotos/reduced_5.mat4.jpg}
%\centering\textit{Foto do Circuito 2.}
%\vspace{2cm}
\end{minipage}}
\caption{Forma de onda do Circuito 4.}
\end{figure}

\[
t_{\mathrm{delay,4}} = \underline{\hspace{2cm}}~\mathrm{ms}
\]

\subsubsection{Circuito 5: Diodo com filtro capacitivo (meia onda filtrada)}

Com a inclusão de um capacitor em paralelo ao resistor, observa-se um processo de carga rápida durante a condução e descarga lenta entre os picos, produzindo tensão suavizada e ondulação (\textit{ripple}) reduzida.

%\textbf{Espaço reservado para foto:}
\begin{figure}[h]
\centering
\fbox{\begin{minipage}{0.9\linewidth}
%\vspace{2cm}
\includegraphics[scale=0.133]{fotos/reduced_4.mat4.jpg}
%\centering\textit{Foto do Circuito 2.}
%\vspace{2cm}
\end{minipage}}
\caption{Forma de onda do Circuito 5.}
\end{figure}

\[
t_{\mathrm{delay,5}} = \underline{\hspace{2cm}}~\mathrm{ms}
\]

Para esse circuito, cabe observação quanto à tensão do pico superior, que apresenta uma distorção, quando, na verdade, esperava-se um espelhamento do semi-ciclo inferior. Conforme discutido em laboratório, isso ocorre devido ao descasamento de impedância entre o osciloscópio e a carga representada pelo circuito medido, o que propicia reflexões de ondas eletromagnéticas no circuito e, consequentemente, a distorção observada.

%\noindent\textit{B. Análise de Circuitos com diodos BY127.}

%\hspace{1cm}Durante a execução da Parte B, foram obtidas as formas de onda de tensão de entrada e de saída para cada circuito analisado. Os valores medidos de atenuação e defasagem, observados nas Figuras 4 a 8, foram registrados e estão sumarizados na Tabela II.

%% \vspace{-.25cm}

%\begin{figure}[!h]
%    \centering
%    \includegraphics[width=7cm]{fotos/1.jpg}
%    \caption{Circuito 1}
%    \label{fig:circ1}
%\end{figure}

%% \vspace{-.25cm}

%\begin{figure}[!h]
%    \centering
%    \includegraphics[width=7cm]{fotos/1.jpg}
%    \caption{Circuito 2}
%    \label{fig:circ2}
%\end{figure}

% \vspace{-.25cm}

%\begin{figure}[!h]
%    \centering
%    \includegraphics[width=7cm]{fotos/1.jpg}
%    \caption{Circuito 3}
%    \label{fig:circ3}
%\end{figure}

% \vspace{-.25cm}

%\begin{figure}[!h]
%    \centering
%    \includegraphics[width=7cm]{fotos/1.jpg}
%    \caption{Circuito 4}
%    \label{fig:circ4}
%\end{figure}

% \vspace{-.25cm}

%\begin{figure}[!h]
%    \centering
%    \includegraphics[width=7cm]{fotos/1.jpg}
%    \caption{Circuito 5}
%    \label{fig:circ5}
%\end{figure}

% \vspace{-.25cm}

De forma resumida, a Tabela~\ref{tab:defasagem} apresenta os valores de defasagem e atenuação medidos para cada circuito analisado.

\begin{table}[!h]
    \centering
    \caption{Defasagem e Atenuação dos Circuitos}
    \label{tab:defasagem}
    \vspace{0.25cm}
    \begin{tabular}{ccc}
        \hline
        \rule{0pt}{3ex}\textbf{Circuito} & \textbf{Defasagem} [ms] & \textbf{Atenuação} [V]\\[5pt]
        \hline
        \rule{0pt}{3ex} 1 &  & \\
        2 &  & \\
        3 & - & - \\
        4 &  & \\
        5 &  & \\[5pt]
        \hline
    \end{tabular}
\end{table}

%\section{ANALISE DOS RESULTADOS EXPERIMENTAIS}
\subsection{Análise dos Resultados Experimentais}

\noindent\textit{A. Curva característica do diodo BY127.}

\hspace{1cm}A análise dos valores apresentados na Tabela~\ref{tab:direta} e da curva mostrada na Figura~\ref{fig:curvaIV} evidencia o comportamento típico do diodo BY127 sob diferentes condições de polarização. 

\hspace{1cm}Observa-se que, quando submetido à polarização direta, o dispositivo inicia a condução de corrente de forma significativa apenas para tensões superiores a aproximadamente $0.6 \, V$. Por outro lado, para valores de $V_D$ inferiores a zero, correspondentes à polarização reversa, não foi registrada condução apreciável, mantendo-se a corrente praticamente nula.

\hspace{1cm}Esse comportamento está em consonância com as especificações presentes no datasheet do BY127, que indicam tensão de limiar entre $0.6 \, V$ e $0.8 \, V$ e corrente desprezível na região de polarização reversa. Assim, os resultados obtidos confirmam o regime de operação esperado para esse tipo de diodo retificador.

\noindent\textit{B. Análise de Circuitos com diodos BY127.}

\hspace{1cm}Os resultados referentes à segunda parte do experimento, sintetizados na Tabela~\ref{tab:defasagem}, permitem observar que o diodo real pode ser interpretado como um diodo ideal acrescido de uma queda de tensão em série. Essa característica manifesta-se diretamente na atenuação verificada entre as tensões de entrada e saída, evidenciando o comportamento não ideal do componente.

\hspace{1cm}A Figura~\ref{} ilustra o funcionamento típico de um retificador de meia onda: a condução ocorre apenas durante a semiciclo positivo, enquanto a polarização reversa do diodo bloqueia a passagem de corrente no semiciclo negativo. Já a Figura~\ref{} apresenta maior atenuação relativa, resultado da utilização de dois diodos em série, cada um contribuindo com sua respectiva queda de tensão.

\hspace{1cm}Na configuração ilustrada na Figura~\ref{}, os diodos encontram-se polarizados de forma oposta, o que impede a condução em ambos os semiciclos e faz com que o circuito se comporte como um circuito aberto, independentemente da tensão aplicada. Em contraste, no caso mostrado na Figura 8, os diodos estão dispostos em paralelo, de modo que o caminho de condução depende da polaridade da fonte: o diodo superior conduz na polarização direta e o inferior conduz na polarização reversa.

\hspace{1cm}Por fim, o circuito correspondente à Figura~\ref{} representa um retificador com capacitor de filtro, também conhecido como retificador de pico. Nesse arranjo, considerando $V_{in}$, $V_{out}$ e $V_t$ como as tensões de entrada, saída e queda no diodo, respectivamente, observam-se dois regimes distintos. Inicialmente, quando $V_{in} > V_{out}$, o diodo conduz e o capacitor carrega até o valor máximo da onda de entrada. Em seguida, com o diodo despolarizado, o capacitor passa a fornecer energia à carga, produzindo uma descarga exponencial dada por $V_{out} = V_{in} \, e^{-t/RC}$. Quando a tensão de entrada volta a superar a tensão no capacitor, o ciclo de carga se reinicia.

%\subsection{Discussão Geral}

Em síntese, os resultados experimentais confirmam o comportamento teórico do diodo semicondutor. A curva direta apresentou limiar próximo de \(0{,}5~\mathrm{V}\), enquanto a curva reversa manteve corrente praticamente nula, evidenciando o adequado bloqueio reverso. Nos circuitos retificadores, cada topologia apresentou características condizentes com sua função prevista, validando os conceitos de condução seletiva, recorte de semiciclo, e filtragem capacitiva.
