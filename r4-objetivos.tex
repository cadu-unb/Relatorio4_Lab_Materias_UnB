\section{Objetivos}

%\hspace{1cm} Levantar a curva característica $I_D\times V_D$ para o diodo BY127 e analisar a curva de resposta para os modelos de circuito com esse diodo.

Os objetivos deste experimento são estabelecer a caracterização elétrica do diodo semicondutor do tipo BY127 e analisar seu comportamento em diferentes condições de operação. De forma específica, o experimento visa:

\begin{itemize}
    \item Levantar a curva característica \( I_D \times V_D \) do diodo em polarização direta, identificando a tensão limiar de condução e o comportamento exponencial previsto pela equação de Shockley;
    \item Determinar a característica \( I_D \times V_D \) em polarização reversa, verificando a ausência de condução e o regime de corrente de saturação abaixo da tensão de ruptura;
    \item Investigar a resposta dinâmica do diodo quando submetido a um sinal senoidal, analisando cinco configurações distintas de circuitos retificadores;
    \item Observar e interpretar as formas de onda medidas no resistor de carga por meio do osciloscópio, relacionando-as com o funcionamento das topologias de retificação;
    \item Desenvolver competências práticas de montagem de circuitos, operação de instrumentos de medição e análise crítica de dados experimentais.
\end{itemize}

Esses objetivos buscam integrar fundamentos teóricos sobre junções PN, dispositivos semicondutores e retificação, com a prática laboratorial necessária para compreensão aprofundada do comportamento do diodo BY127.

\vspace{.25cm}
