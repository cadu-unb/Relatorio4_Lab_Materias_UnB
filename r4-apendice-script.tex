\section{Script Python de plotagem dos dados da tabela}
\label{app:hall}
\begin{lstlisting}[language=Python, caption={Levantamento dos pontos da curva característica do diodo}, label={lst:hall}, basicstyle=\ttfamily\small, frame=single, breaklines=true, numbers=left, numberstyle=\tiny, keywordstyle=\color{blue}, commentstyle=\color{green!50!black}, stringstyle=\color{orange}]
import numpy as np
import matplotlib.pyplot as plt

# Dados experimentais do procedimento 2a1 (polarizacao direta)
# Tensao aplicada (V)
Vd = np.array([0.10, 0.20, 0.30, 0.40,
               0.50, 0.60, 0.70, 0.71])

# Corrente medida (mA) -> conversao para A
Id_mA = np.array([0.0, 0.0, 0.0, 0.0,
                  0.7, 3.3, 29.7, 30.0])
Id = Id_mA / 1000.0  # Converte de mA para A

# Plot da curva I x V
plt.figure(figsize=(8,5))

# Pontos experimentais
plt.scatter(Vd, Id, color='blue', label='Dados experimentais')

# Linhas conectando os pontos
plt.plot(Vd, Id, color='red', linestyle='-', linewidth=1.2, label='Conexao entre medidas consecutivas')

# Rotulos junto aos pontos
for x, y in zip(Vd, Id):
    if y==0.03:
        break;
    plt.text(x + 0.005, y+0.001, f'({x:.2f} V, {y*1000:.1f} mA)',
             fontsize=8, color='black', rotation=30)
plt.text(x + 0.005, y-0.001, f'({0.71:.2f} V, {0.03*1000:.1f} mA)',
             fontsize=8, color='black', rotation=30)

plt.xlabel('Tensao no diodo $V_D$ (V)')
plt.ylabel('Corrente no diodo $I_D$ (A)')
plt.title('Curva Caracteristica do Diodo BY127 (Polarizacao Direta)')
plt.grid(True)
plt.legend()
plt.tight_layout()
plt.show()

# Exibir valores no terminal
print("Dados experimentais utilizados (V_D, I_D):")
for x, y in zip(Vd, Id):
    print(f"{x:.2f} V  ->  {y*1000:.1f} mA")
\end{lstlisting}