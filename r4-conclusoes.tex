\section{Conclusões}
%\section{Conclusão}
%\hspace{1cm}Os resultados obtidos e a análise subsequente permitiram compreender de forma abrangente o comportamento do diodo BY127 tanto em regime de polarização direta quanto reversa. Evidenciou-se, ainda, a versatilidade do componente em aplicações de circuitos eletrônicos, especialmente no contexto de retificação e condicionamento de sinais. A comparação entre os dados experimentais e a teoria consolidada mostrou-se coerente, reforçando a confiabilidade do procedimento adotado e a adequação dos modelos teóricos empregados.

O experimento permitiu a análise detalhada do comportamento elétrico do diodo semicondutor BY127, tanto no regime estático (\textit{curva característica} $I_D \times V_D$) quanto no regime dinâmico, quando submetido a sinal senoidal em diversas configurações de retificação. Os resultados obtidos na polarização direta evidenciaram uma tensão limiar de condução próxima de \(0{,}5~\mathrm{V}\), valor típico para diodos de silício, enquanto a polarização reversa apresentou corrente praticamente nula em toda a faixa de tensões aplicada, confirmando a elevada capacidade de bloqueio reverso do dispositivo.

A análise das cinco montagens retificadoras demonstrou com clareza as funcionalidades individuais de cada topologia. O circuito de meia onda (Circuito 1) confirmou o comportamento de condução unilateral, enquanto a inclusão de dois diodos em série (Circuito 2) evidenciou o aumento da tensão mínima necessária para início da condução. A configuração antissérie (Circuito 3) mostrou-se eficaz no bloqueio mútuo dos semiciclos, resultando em condução mínima. Por outro lado, o circuito em antiparalelo (Circuito 4) permitiu a condução alternada, produzindo uma forma de onda bidirecional limitada pela queda direta de cada diodo. Finalmente, a configuração com filtro capacitivo (Circuito 5) demonstrou uma significativa redução da ondulação (\textit{ripple}), evidenciando o processo de carga e descarga do capacitor e sua importância na suavização da tensão retificada.

De forma geral, os objetivos propostos foram plenamente alcançados. Os resultados experimentais mostraram-se coerentes com os modelos teóricos estudados, e permitiram observar fenômenos essenciais à eletrônica de potência e à conversão AC--DC. A atividade também contribuiu para o desenvolvimento de habilidades práticas, incluindo a montagem de circuitos, uso de instrumentos de medição e análise crítica de dados.

Como sugestões para aprimoramentos futuros, recomenda-se: (i) investigar o comportamento do diodo sob sinais de diferentes frequências; (ii) comparar o desempenho de diferentes tipos de diodos (retificadores, zener, Schottky); (iii) utilizar capacitores de distintos valores para análise quantitativa do \textit{ripple}; e (iv) explorar medições adicionais, como tempo de recuperação reversa e potência dissipada.

Os conhecimentos adquiridos neste experimento constituem uma base sólida para estudos avançados envolvendo dispositivos semicondutores e circuitos retificadores, evidenciando a relevância prática e teórica do tema no contexto da engenharia eletrônica.
