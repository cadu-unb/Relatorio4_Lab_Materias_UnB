\documentclass[10pt,twocolumn,letterpaper]{article}

% Pacotes basicos - maxima compatibilidade Windows
\usepackage{geometry}
\usepackage{times}
\usepackage{titlesec}
\usepackage{url}
\usepackage{graphicx,xcolor,comment,enumerate,multirow,multicol} 
\usepackage{amsmath,amsthm,amsfonts,amssymb,dsfont,mathtools, array}

\usepackage{enumitem}
\setlist[itemize]{
    itemsep=2pt,        % Espaçamento vertical entre itens
    parsep=0pt,         % Espaçamento entre parágrafos dentro de um item
    topsep=0pt,         % Espaçamento vertical antes do primeiro item da lista
    partopsep=0pt,      % Espaçamento extra quando a lista começa no início de um parágrafo
    leftmargin=1.5em    % Opcional: Ajusta a margem esquerda (se estiver muito indentado)
}

% Configuracao da pagina
\geometry{
    letterpaper,
    left=0.75in,
    right=0.75in,
    top=0.75in,
    bottom=1in
}

%% Edição confortável
% Inclui o pacote xcolor com a opção para nomes de cores
\usepackage[svgnames]{xcolor}
% Para desativar comente a linha utiliando %
%% Define a cor do texto usando o código hexadecimal
\definecolor{Cornsilk}{HTML}{FFF8DC}

% Define a cor de fundo da página como preta
\pagecolor{Black}

% Define a cor padrão do texto para todo o documento
\color{white}

% Configuracao das secoes
\titleformat{\section}[block]
{\normalfont\fontsize{10}{12}\bfseries}
{\thesection.}{0.5em}{}

% Remove numeracao das paginas
\pagestyle{empty}

% Configuracoes de espacamento
\setlength{\columnsep}{0.25in}
\setlength{\parindent}{0pt}
\setlength{\parskip}{6pt}

\begin{document}

% Titulo centralizado em coluna unica
\twocolumn[
\begin{center}
    {\fontsize{16}{19}\selectfont\bfseries 
    Experimento 4 - Diodo semicondutor}

    \vspace{1cm}
    
    {\fontsize{11}{13}\selectfont 
     Carlos Eduardo da S. Papa – 232013390, Ronan Cunha Freitas – 232013425 }

    \vspace{0.35cm}   

    {\fontsize{11}{13}\selectfont 
    Turma 02}
    
    \vspace{1cm}  
\end{center}
]

\section{OBJETIVOS}

\hspace{1cm} O presente experimento 

\section{MATERIAIS e EQUIPAMENTO UTILIZADOS}

\begin{itemize}
    \item aaaa;
    \item aaaa;
    \item aaaa;
\end{itemize}

% \vspace{.75cm}

\section{PROCEDIMENTOS EXPERIMENTAIS}

\hspace{1cm} Inicialmente, 

\begin{figure}[h]
    \centering
    \includegraphics[width=7cm]{Imagem1.png}
    \caption{Montagem do experimento}
    \label{fig:montagem}
\end{figure}

\hspace{1cm} Após 

\section{RESULTADOS EXPERIMENTAIS}

\hspace{1cm} A partir ...

\vspace{-.25cm}

\begin{table}[htbp]
    \centering
    \caption{Medições de Tensão em Função da Corrente DC}
    \label{tab:medicoes_tensao}
    \vspace{0.25cm}
    \begin{tabular}{ccc}
        \hline
        \rule{0pt}{3ex}\textbf{Faixa Monocromática} & \textbf{Ângulo} [°] & \textbf{Tensão} [V]\\[5pt]
        \hline
        \rule{0pt}{3ex}Azul 1 & 12 & 0.65 \\
        Azul 2 & 14 & 0.52 \\
        Azul 3 & 16 & 0.51 \\
        Verde & 20 & 0.32 \\
        Laranja & 21 & 0.20 \\[5pt]
        \hline
    \end{tabular}
\end{table}

\section{ANALISE DOS RESULTADOS EXPERIMENTAIS}

\noindent\textit{A. ...}

\noindent\textit{B. ...}

\section{Conclusão}



\section{REFERENCIAS BIBLIOGRAFICAS}

{\small
\begin{enumerate}

    \item CESCHIN, Artemis M. Apostila de materiais eletricos e magneticos.

    \item REZENDE, Sergio M. Materiais e Dispositivos Eletrônicos. 2ª ed. São Paulo: Editora Livraria da Física, 2004.

    \item HAYT, W. H. Jr. Eletromagnetismo. 6ª ed. Rio de Janeiro: LTC, 1995.
    
\end{enumerate}
}

\end{document}